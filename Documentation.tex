% Options for packages loaded elsewhere
\PassOptionsToPackage{unicode}{hyperref}
\PassOptionsToPackage{hyphens}{url}
%
\documentclass[
]{article}
\usepackage{amsmath,amssymb}
\usepackage{iftex}
\ifPDFTeX
  \usepackage[T1]{fontenc}
  \usepackage[utf8]{inputenc}
  \usepackage{textcomp} % provide euro and other symbols
\else % if luatex or xetex
  \usepackage{unicode-math} % this also loads fontspec
  \defaultfontfeatures{Scale=MatchLowercase}
  \defaultfontfeatures[\rmfamily]{Ligatures=TeX,Scale=1}
\fi
\usepackage{lmodern}
\ifPDFTeX\else
  % xetex/luatex font selection
\fi
% Use upquote if available, for straight quotes in verbatim environments
\IfFileExists{upquote.sty}{\usepackage{upquote}}{}
\IfFileExists{microtype.sty}{% use microtype if available
  \usepackage[]{microtype}
  \UseMicrotypeSet[protrusion]{basicmath} % disable protrusion for tt fonts
}{}
\makeatletter
\@ifundefined{KOMAClassName}{% if non-KOMA class
  \IfFileExists{parskip.sty}{%
    \usepackage{parskip}
  }{% else
    \setlength{\parindent}{0pt}
    \setlength{\parskip}{6pt plus 2pt minus 1pt}}
}{% if KOMA class
  \KOMAoptions{parskip=half}}
\makeatother
\usepackage{xcolor}
\usepackage[margin=1in]{geometry}
\usepackage{color}
\usepackage{fancyvrb}
\newcommand{\VerbBar}{|}
\newcommand{\VERB}{\Verb[commandchars=\\\{\}]}
\DefineVerbatimEnvironment{Highlighting}{Verbatim}{commandchars=\\\{\}}
% Add ',fontsize=\small' for more characters per line
\usepackage{framed}
\definecolor{shadecolor}{RGB}{248,248,248}
\newenvironment{Shaded}{\begin{snugshade}}{\end{snugshade}}
\newcommand{\AlertTok}[1]{\textcolor[rgb]{0.94,0.16,0.16}{#1}}
\newcommand{\AnnotationTok}[1]{\textcolor[rgb]{0.56,0.35,0.01}{\textbf{\textit{#1}}}}
\newcommand{\AttributeTok}[1]{\textcolor[rgb]{0.13,0.29,0.53}{#1}}
\newcommand{\BaseNTok}[1]{\textcolor[rgb]{0.00,0.00,0.81}{#1}}
\newcommand{\BuiltInTok}[1]{#1}
\newcommand{\CharTok}[1]{\textcolor[rgb]{0.31,0.60,0.02}{#1}}
\newcommand{\CommentTok}[1]{\textcolor[rgb]{0.56,0.35,0.01}{\textit{#1}}}
\newcommand{\CommentVarTok}[1]{\textcolor[rgb]{0.56,0.35,0.01}{\textbf{\textit{#1}}}}
\newcommand{\ConstantTok}[1]{\textcolor[rgb]{0.56,0.35,0.01}{#1}}
\newcommand{\ControlFlowTok}[1]{\textcolor[rgb]{0.13,0.29,0.53}{\textbf{#1}}}
\newcommand{\DataTypeTok}[1]{\textcolor[rgb]{0.13,0.29,0.53}{#1}}
\newcommand{\DecValTok}[1]{\textcolor[rgb]{0.00,0.00,0.81}{#1}}
\newcommand{\DocumentationTok}[1]{\textcolor[rgb]{0.56,0.35,0.01}{\textbf{\textit{#1}}}}
\newcommand{\ErrorTok}[1]{\textcolor[rgb]{0.64,0.00,0.00}{\textbf{#1}}}
\newcommand{\ExtensionTok}[1]{#1}
\newcommand{\FloatTok}[1]{\textcolor[rgb]{0.00,0.00,0.81}{#1}}
\newcommand{\FunctionTok}[1]{\textcolor[rgb]{0.13,0.29,0.53}{\textbf{#1}}}
\newcommand{\ImportTok}[1]{#1}
\newcommand{\InformationTok}[1]{\textcolor[rgb]{0.56,0.35,0.01}{\textbf{\textit{#1}}}}
\newcommand{\KeywordTok}[1]{\textcolor[rgb]{0.13,0.29,0.53}{\textbf{#1}}}
\newcommand{\NormalTok}[1]{#1}
\newcommand{\OperatorTok}[1]{\textcolor[rgb]{0.81,0.36,0.00}{\textbf{#1}}}
\newcommand{\OtherTok}[1]{\textcolor[rgb]{0.56,0.35,0.01}{#1}}
\newcommand{\PreprocessorTok}[1]{\textcolor[rgb]{0.56,0.35,0.01}{\textit{#1}}}
\newcommand{\RegionMarkerTok}[1]{#1}
\newcommand{\SpecialCharTok}[1]{\textcolor[rgb]{0.81,0.36,0.00}{\textbf{#1}}}
\newcommand{\SpecialStringTok}[1]{\textcolor[rgb]{0.31,0.60,0.02}{#1}}
\newcommand{\StringTok}[1]{\textcolor[rgb]{0.31,0.60,0.02}{#1}}
\newcommand{\VariableTok}[1]{\textcolor[rgb]{0.00,0.00,0.00}{#1}}
\newcommand{\VerbatimStringTok}[1]{\textcolor[rgb]{0.31,0.60,0.02}{#1}}
\newcommand{\WarningTok}[1]{\textcolor[rgb]{0.56,0.35,0.01}{\textbf{\textit{#1}}}}
\usepackage{graphicx}
\makeatletter
\def\maxwidth{\ifdim\Gin@nat@width>\linewidth\linewidth\else\Gin@nat@width\fi}
\def\maxheight{\ifdim\Gin@nat@height>\textheight\textheight\else\Gin@nat@height\fi}
\makeatother
% Scale images if necessary, so that they will not overflow the page
% margins by default, and it is still possible to overwrite the defaults
% using explicit options in \includegraphics[width, height, ...]{}
\setkeys{Gin}{width=\maxwidth,height=\maxheight,keepaspectratio}
% Set default figure placement to htbp
\makeatletter
\def\fps@figure{htbp}
\makeatother
\setlength{\emergencystretch}{3em} % prevent overfull lines
\providecommand{\tightlist}{%
  \setlength{\itemsep}{0pt}\setlength{\parskip}{0pt}}
\setcounter{secnumdepth}{-\maxdimen} % remove section numbering
\ifLuaTeX
  \usepackage{selnolig}  % disable illegal ligatures
\fi
\IfFileExists{bookmark.sty}{\usepackage{bookmark}}{\usepackage{hyperref}}
\IfFileExists{xurl.sty}{\usepackage{xurl}}{} % add URL line breaks if available
\urlstyle{same}
\hypersetup{
  hidelinks,
  pdfcreator={LaTeX via pandoc}}

\author{}
\date{\vspace{-2.5em}}

\begin{document}

\begin{Shaded}
\begin{Highlighting}[]
\CommentTok{{-}{-}{-}}
\AnnotationTok{title:}\CommentTok{ "Guide to Social Media Analysis Functions"}
\AnnotationTok{author:}\CommentTok{ "A. Rahman"}
\CommentTok{{-}{-}{-}}

\FunctionTok{\# Introduction}

\NormalTok{This document is a comprehensive guide that demonstrates the usage of functions provided by the SocialNetAnalyser package for social media data analysis.}

\NormalTok{When you knit this document, it will include the output from the R code chunks that invoke functions to calculate metrics, plot engagement trends, perform sentiment analysis, and visualize posting trends. This guide assumes that }\InformationTok{\textasciigrave{}social\_data\textasciigrave{}}\NormalTok{ is a preloaded data frame structured for social media analysis.}

\FunctionTok{\# Preliminaries}

\NormalTok{Before proceeding, we need to set up our environment by loading the necessary functions and data.}

\FunctionTok{\#\# Load Required Functions and Data}




\InformationTok{\textasciigrave{}\textasciigrave{}\textasciigrave{}r}
\CommentTok{\# Loading functions from the SocialNetAnalyser package}
\FunctionTok{source}\NormalTok{(}\StringTok{"\textasciitilde{}/Downloads/SocialNetAnalyser/R/main.R"}\NormalTok{)}
\end{Highlighting}
\end{Shaded}

\begin{verbatim}
##   total_posts total_likes total_comments total_shares average_engagement_rate
## 1          29        1670            142           90                65.58621
\end{verbatim}

\includegraphics{Documentation_files/figure-latex/load-functions-1.pdf}

\begin{verbatim}
##          date sentiment
## 1  2022-01-01  Negative
## 2  2022-01-02  Positive
## 3  2022-01-03  Positive
## 4  2022-01-04   Neutral
## 5  2022-01-05   Neutral
## 6  2022-01-06  Negative
## 7  2022-01-07  Positive
## 8  2022-01-08  Positive
## 9  2022-01-09   Neutral
## 10 2022-01-10   Neutral
\end{verbatim}

\includegraphics{Documentation_files/figure-latex/load-functions-2.pdf}

\begin{Shaded}
\begin{Highlighting}[]
\CommentTok{\# Loading the dataset}
\FunctionTok{load}\NormalTok{(}\StringTok{"data/social\_data.RData"}\NormalTok{)}
\end{Highlighting}
\end{Shaded}

\hypertarget{inspect-data-structure}{%
\subsection{Inspect Data Structure}\label{inspect-data-structure}}

\begin{Shaded}
\begin{Highlighting}[]
\CommentTok{\# It\textquotesingle{}s important to understand the structure of our data before analysis}
\FunctionTok{str}\NormalTok{(social\_data)}
\end{Highlighting}
\end{Shaded}

\begin{verbatim}
## 'data.frame':    10 obs. of  5 variables:
##  $ date    : Date, format: "2022-01-01" "2022-01-02" ...
##  $ posts   : num  2 3 2 4 5 3 4 2 1 3
##  $ likes   : num  100 150 120 300 230 200 180 140 90 160
##  $ comments: num  10 15 12 20 18 25 10 14 8 10
##  $ shares  : num  5 10 7 15 10 20 5 7 3 8
\end{verbatim}

\hypertarget{data-analysis}{%
\section{Data Analysis}\label{data-analysis}}

With our data loaded and understood, we can now proceed to analyze it
using various functions.

\hypertarget{calculate-metrics}{%
\subsection{Calculate Metrics}\label{calculate-metrics}}

Calculate and summarize the basic engagement metrics from the social
media data.

\begin{Shaded}
\begin{Highlighting}[]
\NormalTok{metrics\_table }\OtherTok{\textless{}{-}} \FunctionTok{calculate\_metrics}\NormalTok{(social\_data)}
\NormalTok{metrics\_table}
\end{Highlighting}
\end{Shaded}

\begin{verbatim}
##   total_posts total_likes total_comments total_shares average_engagement_rate
## 1          29        1670            142           90                65.58621
\end{verbatim}

\hypertarget{engagement-trends-visualization}{%
\subsection{Engagement Trends
Visualization}\label{engagement-trends-visualization}}

Visualize the engagement trends over time using the number of likes,
comments, and shares.

\begin{Shaded}
\begin{Highlighting}[]
\NormalTok{engagement\_plot }\OtherTok{\textless{}{-}} \FunctionTok{plot\_engagement}\NormalTok{(social\_data)}
\FunctionTok{print}\NormalTok{(engagement\_plot)}
\end{Highlighting}
\end{Shaded}

\includegraphics{Documentation_files/figure-latex/engagement-plot-1.pdf}

\hypertarget{sentiment-analysis}{%
\subsection{Sentiment Analysis}\label{sentiment-analysis}}

Assign a random sentiment score to each social media post and display
the results.

\begin{Shaded}
\begin{Highlighting}[]
\NormalTok{sentiment\_results }\OtherTok{\textless{}{-}} \FunctionTok{analyse\_sentiment}\NormalTok{(social\_data)}
\FunctionTok{print}\NormalTok{(sentiment\_results)}
\end{Highlighting}
\end{Shaded}

\begin{verbatim}
##          date sentiment
## 1  2022-01-01  Positive
## 2  2022-01-02  Positive
## 3  2022-01-03   Neutral
## 4  2022-01-04  Positive
## 5  2022-01-05  Positive
## 6  2022-01-06  Positive
## 7  2022-01-07  Positive
## 8  2022-01-08   Neutral
## 9  2022-01-09  Positive
## 10 2022-01-10  Positive
\end{verbatim}

\hypertarget{posting-trends-visualization}{%
\subsection{Posting Trends
Visualization}\label{posting-trends-visualization}}

Plot the trends showing the number of posts over time.

\begin{Shaded}
\begin{Highlighting}[]
\NormalTok{post\_trend\_plot }\OtherTok{\textless{}{-}} \FunctionTok{plot\_post\_trends}\NormalTok{(social\_data)}
\FunctionTok{print}\NormalTok{(post\_trend\_plot)}
\end{Highlighting}
\end{Shaded}

\includegraphics{Documentation_files/figure-latex/post-trends-1.pdf}

\hypertarget{conclusion}{%
\section{Conclusion}\label{conclusion}}

The SocialNetAnalyser package provides a suite of tools that makes the
analysis of social media data both accessible and insightful. By
employing these functions, we can derive meaningful patterns and metrics
from social data.

\end{document}
